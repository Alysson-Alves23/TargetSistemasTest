\documentclass{article}
\usepackage{amsmath}

\begin{document}

\title{Problema de Encontro entre Veículos}
\author{}
\date{}
\maketitle

\section*{Descrição do Problema}
Dois veículos, um carro e um caminhão, saem respectivamente de cidades opostas pela mesma rodovia. O carro, de Ribeirão Preto em direção a Barretos, a uma velocidade constante de 90 km/h, e o caminhão, de Barretos em direção a Ribeirão Preto, a uma velocidade constante de 80 km/h. Quando eles se cruzarem no percurso, qual estará mais próximo da cidade de Ribeirão Preto?

\begin{itemize}
    \item Distância entre Ribeirão Preto e Barretos: 125 km.
    \item Velocidade do carro: 90 km/h.
    \item Velocidade do caminhão: 80 km/h.
    \item O carro leva 5 minutos a mais para passar em cada um dos 3 pedágios, totalizando 15 minutos de atraso.
\end{itemize}

\section*{Passo 1: Ajustar a Velocidade do Carro}
O carro perde 15 minutos (0,25 horas) devido aos pedágios. A velocidade média efetiva do carro pode ser ajustada da seguinte forma:

\[
v_{\text{carro}} = \frac{90 \text{ km}}{1 \text{ hora} + 0.25 \text{ horas}} = \frac{90}{1.25} = 72 \text{ km/h}
\]

Agora, a velocidade do carro, considerando os pedágios, é de 72 km/h.

\section*{Passo 2: Calcular o Tempo até o Encontro}
A velocidade relativa entre o carro e o caminhão é a soma de suas velocidades:

\[
v_{\text{relativa}} = 72 \text{ km/h} + 80 \text{ km/h} = 152 \text{ km/h}
\]

O tempo necessário para que os veículos se encontrem é:

\[
t = \frac{125 \text{ km}}{152 \text{ km/h}} \approx 0.822 \text{ horas} \approx 49.32 \text{ minutos}
\]

\section*{Passo 3: Determinar a Distância Percorrida até o Ponto de Encontro}
- \textbf{Distância percorrida pelo carro}:

\[
d_{\text{carro}} = 72 \text{ km/h} \times 0.822 \text{ horas} \approx 59.78 \text{ km}
\]

- \textbf{Distância percorrida pelo caminhão}:

\[
d_{\text{caminhao}} = 80 \text{ km/h} \times 0.822 \text{ horas} \approx 65.76 \text{ km}
\]

\section*{Passo 4: Qual Veículo Estará Mais Próximo de Ribeirão Preto?}
No momento em que eles se cruzam:
- O carro percorreu 59.78 km a partir de Ribeirão Preto, então ele está a:

\[
 59.78 \text{ km} 
\]

- O caminhão percorreu 65.76 km a partir de Barretos, então ele está a:

\[
125 \text{ km} - 65.76 \text{ km} \approx 59.24 \text{ km} \text{ de Ribeirão Preto}
\]

\section*{Conclusão}
No momento em que eles se cruzarem, \textbf{o caminhão estará mais próximo de Ribeirão Preto}, a uma distância de aproximadamente \(59.24 km\), enquanto o carro estará a \(59.78 km\)

\end{document}
